\documentclass[12pt]{article}
\usepackage[utf8]{inputenc}
\usepackage[T1]{fontenc}
\usepackage{amsmath}
\usepackage{amssymb}
\usepackage{amsthm}
\usepackage{geometry}
\geometry{margin=1in}
\usepackage{setspace}
\usepackage{hyperref}
\usepackage{booktabs}
\usepackage{graphicx}
\doublespacing

\newtheorem{proposition}{Proposition}
\newtheorem{definition}{Definition}

% Math macros
\newcommand{\kB}{k_{\mathrm{B}}}
\newcommand{\Teff}{T_{\mathrm{eff}}}
\newcommand{\Deff}{D_{\mathrm{eff}}}

\title{Coherence time in neural oscillator assemblies sets the speed of thought}

\author{Ian Todd\\
Sydney Medical School, University of Sydney\\
Sydney, NSW, Australia\\
\texttt{itod2305@uni.sydney.edu.au}}

\date{\today}

\begin{document}

\maketitle

\begin{abstract}
Neuroscience has established that cognitive processing depends on coherent oscillations across neural assemblies: working memory maintenance requires sustained theta-gamma coupling, attention modulates inter-areal synchronization, and perceptual binding emerges from transient phase alignment. Yet the physical principles determining how fast these assemblies can synchronize---and thus how fast we can think---remain incompletely formalized. We derive a quantitative framework showing that coherence time in coupled oscillator networks scales exponentially with coordination depth. For $M$ semi-independent modules requiring phase alignment within tolerance $\varepsilon$ at Kuramoto coherence $r$ and phase-exploration rate $\Delta\omega$:
\[
\tau_{\mathrm{coh}} = \frac{1}{\Delta\omega}\left(\frac{2\pi}{\varepsilon}\right)^{\alpha(1-r)(M-1)}
\]
where circular variance $(1-r)$ governs phase dispersion and $\alpha$ depends on network topology. All topologies exhibit positive $M$-scaling under our operational definition; numerical validation in modular networks yields $\hat{\alpha} \approx 0.35$ ($R^2 = 0.71$; Figure~\ref{fig:validation}). Modular networks are the regime where $M$ represents intrinsic semi-independent constraints and the formula has a literal first-passage interpretation; control topologies (all-to-all, sparse) yield fitted $\alpha$ values that reflect collective dynamics rather than coordination. This produces a fundamental speed-flexibility trade-off: increasing coordination depth $M$ expands combinatorial flexibility but slows commits exponentially; tighter coherence (higher $r$) speeds synchronization but restricts dynamics to low-dimensional attractors. The framework explains millisecond-scale perceptual windows, arousal-driven time dilation, alpha frequency correlates of temporal acuity, and metabolic scaling across species.
\end{abstract}

\textbf{Keywords:} neural oscillations, coherence time, dimensional coordination, temporal resolution, working memory, coupled oscillators

\section{Introduction}

\subsection{Neural oscillations determine processing speed}

A fundamental insight from systems neuroscience is that cognition emerges from coherent oscillations across neural assemblies, not merely from individual spike rates. Working memory maintenance requires sustained theta-gamma phase-amplitude coupling~\cite{miller2018,lisman2013}. Attention selectively enhances inter-areal synchronization in gamma band~\cite{fries2015,womelsdorf2007}. Perceptual binding depends on transient phase alignment across sensory cortices~\cite{singer1999,engel2001}. Long-range communication occurs preferentially during coherent states~\cite{varela2001}. The ``communication through coherence'' framework~\cite{fries2005} has become central to understanding neural computation.

Yet despite extensive empirical characterization, the physical principles governing \emph{how fast} distributed assemblies can achieve coherence---and thus how quickly cognitive operations can proceed---remain incompletely formalized. Why does perceptual binding require 30--50 ms rather than 3 ms or 300 ms? Why do larger assemblies integrating more information process more slowly? What determines the relationship between oscillation frequency (e.g., alpha at 10 Hz) and temporal acuity?

\subsection{The missing quantitative framework}

We propose that neural processing speed is fundamentally limited by \emph{coherence time}: the time required for distributed oscillators to achieve sufficient phase alignment for a collective computation to register. This coherence time exhibits three critical properties:

\textbf{1. Exponential scaling with coordination depth.} Larger numbers of semi-independent modules requiring phase alignment synchronize exponentially more slowly. For typical parameters ($r = 0.6$, $\varepsilon = \pi$, $\alpha = 0.6$), increasing $M$ from 5 to 15 slows coherence time by a factor of $\sim 2^4 \approx 16$, while increasing from 10 to 30 yields a $\sim 10^3$-fold slowdown.

\textbf{2. Modulation by coupling strength.} Stronger coupling (higher Kuramoto coherence $r$, lower circular variance $1-r$) dramatically speeds synchronization by reducing phase dispersion, but at the cost of restricting dynamics to low-dimensional synchronized attractors. This creates a fundamental speed-flexibility trade-off.

\textbf{3. Universality across biological oscillators.} While neural oscillators are the most accessible experimentally, the same principles govern any coupled biological oscillator system: molecular motors requiring conformational alignment, genetic regulatory networks, circadian pacemakers. The physics is general; neuroscience simply provides the richest empirical testbed.

\subsection{Empirical puzzles and predictions}

Three phenomena motivate and test our framework:

\textbf{Scale-dependent temporal resolution.} Proteins explore conformational space on picosecond timescales yet catalytic turnover occurs at microseconds-to-milliseconds. Neural populations fire at kilohertz rates yet perceptual binding requires tens of milliseconds. Circadian oscillators maintain 24-hour periods. We demonstrate these span different physical regimes of a unified bound.

\textbf{Subjective time dilation without motor slowing.} Under acute stress (tachypsychia), subjective time slows while reaction times remain constant~\cite{stetson2007}. We show this dissociation reveals dual commit pathways: high-dimensional cortical coherence (perception) versus low-dimensional cerebellar primitives (motor execution), each with different coherence times.

\textbf{Metabolic scaling of temporal acuity.} Critical flicker fusion frequency correlates with mass-specific metabolic rate across three orders of magnitude in body size~\cite{healy2013}. Flies perceive at 240 Hz, humans at 60 Hz, turtles at 15 Hz. We derive this relationship from power-limited coherence time.

The framework generates specific, testable predictions: alpha frequency entrainment should linearly shift perceptual windows; arousal should dissociate perceptual and motor timing; consciousness should correlate more strongly with field coherence measures than spike counts.

\subsection{Structure of this work}

We first derive the coherence time formula from principles of phase synchronization in coupled oscillator networks (§2.1), validate it numerically (§2.2), then embed it in a unified temporal resolution bound combining quantum, noise, and power constraints (§2.3). Section 3 applies the framework to neural systems, demonstrating quantitative agreement with perceptual binding windows, tachypsychia dissociations, and metabolic scaling. Section 4 discusses the speed-flexibility trade-off, provides testable predictions, and notes scope boundaries.

Previous work established sub-Landauer biological computation~\cite{todd2025biosystems} and timing inaccessibility in high-dimensional systems~\cite{todd2025demon}. Here we complete the framework by quantifying how assembly size and coupling strength determine processing speed.

\subsection{Historical context: Biological time and Igamberdiev's framework}

The concept that biological systems exhibit temporal organization distinct from physical time has deep roots. Vernadsky~\cite{vernadsky1945} proposed that living systems possess a non-Euclidean ``biological space-time'' geometry fundamentally different from the Newtonian absolute time of physics. Building on this foundation, Igamberdiev~\cite{igamberdiev1985} formalized the notion of \emph{biological time}---a logarithmic function of physical time that reflects the rate-dependent nature of biological processes.

Igamberdiev's framework remained largely phenomenological, lacking a quantitative mechanistic basis. Our coherence time formula (Eq.~\ref{eq:coh-time}) provides this missing link: the exponential scaling with coordination depth $M$ is precisely the mathematical realization of biological time's logarithmic relationship to physical time. When $M$ modules must achieve phase alignment, coherence time grows as $\tau_{\mathrm{coh}} \propto \exp[\alpha(1-r)(M-1)\ln(2\pi/\varepsilon)]$, making biological time $t_{\mathrm{bio}} \sim \ln(\tau_{\mathrm{coh}}) \propto M$, linear in system complexity.

This echoes Rosen's concept of anticipatory systems~\cite{rosen1985}, where the internal model's timescale must decouple from real-time dynamics to enable prediction. Our formula quantifies how that decoupling emerges from coordination requirements. The framework is compatible with (and provides a mechanistic instantiation of) Igamberdiev's biological time; it does not require a quantum substrate claim.

\section{Theory and Methods}

\subsection{The coherence time formula}

We model biological temporal processing as a sequence of \emph{commits}---thermodynamically irreversible events that register high-dimensional internal state as low-dimensional output. A commit requires: (1) dimensional collapse from distributed state ($\Deff$ dimensions) to registered outcome (few dimensions), (2) dissipation of at least $\kB T \ln 2$ per bit (Landauer bound), and (3) creation of a persistent measurement influencing future dynamics. This thermodynamic registration corresponds to what Matsuno~\cite{matsuno1989} terms an ``internal measurement''---a local irreversible record that distinguishes past from future within the system itself.

By ``commit'' we mean an irreversible registration of distributed state into a low-dimensional outcome that constrains subsequent dynamics---not necessarily a single spike or discrete behavioral choice. Operationally, a commit is the minimal event that can change an externally measurable (or downstream-controller measurable) discrete state. Different tasks have different coordination depths $M$ and different dimensional collapse $\Delta\mathcal{D}$, but the same bottleneck logic applies.

Let $r$ be the mean resultant length (Kuramoto order parameter), and $\kappa(r)$ the von Mises concentration with $r=I_1(\kappa)/I_0(\kappa)$~\cite{mardia2000}.
For small phase window $\varepsilon$ per module, the per-module alignment probability is
\[
p_{\rm align}(\varepsilon,\kappa)\approx \frac{\varepsilon}{2\pi}\frac{e^{\kappa}}{I_0(\kappa)}
\approx \frac{\varepsilon}{2\pi}\sqrt{2\pi\kappa}\quad(\kappa\gg 1).
\]
Assuming independence across modules on the commit timescale, the mean waiting time for all $M$ modules to align is
\begin{equation}\label{eq:coh-mises}
\tau_{\mathrm{coh}} \approx \frac{1}{\Delta\omega}\Big[p_{\rm align}(\varepsilon,\kappa(r))\Big]^{-(M-1)}.
\end{equation}
The exponent is $(M-1)$ because one module serves as the phase reference; only the remaining $M-1$ relative phases must independently fall within tolerance.
To obtain a tractable scaling law, we approximate $p_{\mathrm{align}}$ as a power function of the phase window fraction. In the uncoupled limit ($r \to 0$), the probability is purely geometric: $p_0 \approx \varepsilon/2\pi$. In the perfectly synchronized limit ($r \to 1$), alignment is guaranteed: $p \to 1$. We model the intermediate regime as a reduction in the effective search space---coupling compresses the phase distribution, effectively raising the geometric probability to a power dependent on the disorder $(1-r)$:
\begin{equation}\label{eq:palign-scaling}
p_{\mathrm{align}}(\varepsilon, r) \;\approx\; \left(\frac{\varepsilon}{2\pi}\right)^{\alpha(1-r)}
\end{equation}
where $\alpha$ is a topology-dependent scalar reflecting how efficiently local coupling restricts global phase exploration. Equation~\ref{eq:palign-scaling} is a closure that approximates the concentration-dependent von Mises mass in an $\varepsilon$-window by a single effective exponent; this trades explicit $\kappa(r)$ dependence for a topology-dependent scalar $\alpha$, which we estimate empirically. Substituting into Eq.~\ref{eq:coh-mises} for $M-1$ relative phase constraints yields:
\begin{equation}\label{eq:coh-time}
\boxed{\tau_{\mathrm{coh}} \approx \frac{1}{\Delta\omega}\left(\frac{2\pi}{\varepsilon}\right)^{\alpha(1-r)(M-1)}}
\end{equation}
This derivation clarifies that Eq.~\ref{eq:coh-time} is not an arbitrary fit, but a linear interpolation of the \emph{exponent} of the alignment probability between the incoherent (geometric) and coherent (deterministic) limits. Numerical validation (below) yields $\hat{\alpha} \approx 0.35$ for modular networks.

\textbf{Physical interpretation:} $M$ is the \textbf{coordination depth}---the number of semi-independent modules (locally coherent clusters) whose order parameters must jointly exceed a threshold for a commit to register. $\Delta\omega$ is the \textbf{phase-exploration rate}: the characteristic frequency at which phases mix, combining natural frequency spread and phase diffusion from noise~\cite{ott2008,watanabe1994}. Circular variance $1-r$ quantifies residual phase dispersion~\cite{mardia2000}. Note: $M \le \Deff$ where $\Deff$ is the full manifold dimensionality; $M$ counts only commit-relevant modules.

\textbf{Key assumptions:}
\begin{enumerate}
\item \textbf{Module independence:} Semi-independent dynamics across modules on the commit timescale (weak inter-module coupling compared to intra-module).
\item \textbf{Moderate coupling regime:} Near synchronization onset where topology parameter $\alpha$ is approximately constant; breakdown in deep-synchronization or fully-incoherent limits.
\item \textbf{Stationary noise:} Phase diffusion rate $\Delta\omega$ remains approximately constant during the alignment window $\tau_{\mathrm{coh}}$.
\item \textbf{Modular topology:} The theory applies when modules are internally coherent but not globally phase-locked. Fully synchronized (all-to-all) or highly fragmented (sparse) regimes fall outside its scope.
\end{enumerate}

\subsection{Numerical validation}

To validate Eq.~\ref{eq:coh-time}, we simulated modular Kuramoto networks of $N = 100$ coupled phase oscillators:
\[
d\theta_i = \left[\omega_i + \frac{K}{N}\sum_{j=1}^N A_{ij}\sin(\theta_j - \theta_i)\right]dt + \sigma\,dW_i
\]
where $A_{ij}$ encodes modular topology (strong intra-module coupling $K_{\rm intra} = 1.0$, weak inter-module coupling $K_{\rm inter} = 0.15$), natural frequencies $\omega_i \sim \mathcal{N}(0, \omega_{\rm std})$ with $\omega_{\rm std} = 0.3$ rad/s, and $\sigma = 0.1$ rad/s$^{1/2}$ provides phase noise. Coordination depth $M$ (number of modules) varied from 3 to 10. The phase-exploration rate $\Delta\omega$ was measured dynamically as the standard deviation of instantaneous frequencies across oscillators, accounting for coupling suppression of phase drift.

Coherence time $\tau_{\mathrm{coh}}$ was measured as first passage time to simultaneous phase alignment across all modules (within tolerance $\varepsilon = 2.0$ rad), sustained for dwell time $>0.03$s. We additionally required each module's internal order parameter $r_m \ge r_{\mathrm{threshold}} = 0.6$, ensuring modules are internally coherent before testing inter-module alignment. For each $M$, 20 independent trials. We report median $\tau_{\mathrm{coh}}$ with interquartile range (IQR) to account for heavy-tailed first-passage time distributions.

Fitting $\log(\mathrm{median}\,\tau_{\mathrm{coh}}) = \alpha(1-\bar{r})(M-1)\log(2\pi/\varepsilon) + \mathrm{const}$, where $\bar{r}$ is the mean global order parameter over the post-transient window, yields Table~\ref{tab:topology}.

\begin{table}[ht]
\centering
\caption{Fitted scaling parameters across network topologies. All topologies show positive $M$-scaling in this parameter regime, but the interpretation differs (see text).}
\label{tab:topology}
\begin{tabular}{lccc}
\toprule
Topology & $\bar{r}$ & $\hat{\alpha}$ & $R^2$ \\
\midrule
Modular & 0.94 & 0.35 & 0.71 \\
All-to-all & 0.95 & 0.62 & 0.88 \\
Sparse & 0.79 & 0.26 & 0.96 \\
\bottomrule
\end{tabular}
\end{table}

All topologies show positive scaling in this parameter regime, but the \emph{interpretation} differs. Because $(1-\bar{r})\ln(2\pi/\varepsilon)$ is modest at high coherence ($\bar{r} \approx 0.94$), the exponential appears approximately linear over $M = 3$--$10$; extrapolation to larger $M$ predicts rapid growth. The key observation is that \textbf{modular networks maintain coherence while preserving module independence}---precisely the regime where the first-passage interpretation holds. In all-to-all networks, the module partition is arbitrary (modules are not semi-independent), so $M$ is not an intrinsic coordination depth; the fitted $\alpha$ reflects a surrogate statistic, not the first-passage interpretation. Sparse networks at these parameters are still well-connected; truly fragmented sparse networks ($p \ll 0.03$) would fail to cohere at all. Neural networks are hierarchically modular, operating in the regime where coordination time becomes the limiting factor.

We note that fixing $N=100$ while increasing $M$ reduces individual module size (from $N/M = 33$ at $M=3$ to $N/M = 10$ at $M=10$). In isolated Kuramoto systems, finite-size effects typically \emph{enhance} synchronization speed for smaller populations at this noise level. The fact that $\tau_{\mathrm{coh}}$ increases with $M$ despite the easier internal synchronization of smaller modules confirms that the combinatorial difficulty of inter-module alignment dominates over local finite-size effects.

\begin{figure}[ht]
\centering
\includegraphics[width=\textwidth]{figures/fig_validation_modular_N100.pdf}
\caption{\textbf{Numerical validation of coherence time scaling.} (A) Median coherence time $\tau_{\mathrm{coh}}$ versus coordination depth $M$ for modular Kuramoto networks ($N=100$ oscillators, 20 trials per $M$). Error bars show interquartile range. Dashed line: fitted scaling law with $\hat{\alpha} = 0.35$, $R^2 = 0.71$. (B) Hit fraction (proportion of trials reaching coherence within simulation horizon). All trials succeeded across all $M$ values, indicating the parameter regime is well within the formula's scope.}
\label{fig:validation}
\end{figure}

\begin{figure}[ht]
\centering
\includegraphics[width=\textwidth]{figures/fig_topology_comparison.pdf}
\caption{\textbf{Topology comparison.} Coherence time $\tau_{\mathrm{coh}}$ versus $M$ for (A) modular, (B) all-to-all, and (C) sparse networks. All topologies show positive $M$-scaling (Table~\ref{tab:topology}), but the interpretation differs: modular networks have intrinsic semi-independent modules where $M$ is coordination depth; all-to-all networks lack true module independence, so the fitted $\alpha$ reflects collective dynamics rather than coordination; sparse networks at $p=0.03$ are still well-connected. Error bars: IQR; dashed lines: fitted scaling.}
\label{fig:topology}
\end{figure}

Code available at \url{https://github.com/todd866/coherence-time-biosystems}.

\subsection{The unified temporal resolution bound}

The minimum time between commits is bounded by four physical constraints:

\begin{equation}\label{eq:main-bound}
\boxed{\tau_{\mathrm{eff}} \;=\; \max\left[\tau_{\mathrm{QSL}},\; \tau_{\mathrm{SNR}},\; \tau_{\mathrm{coh}},\; \tau_{\mathrm{power}}\right]}
\end{equation}

where the \textbf{max} operation reflects that the slowest mechanism dominates. The four terms are:

\paragraph{Quantum speed limits ($\tau_{\mathrm{QSL}}$).} The Mandelstam-Tamm and Margolus-Levitin bounds~\cite{mandelstam1945,margolus1998} establish
\[
\tau_{\mathrm{QSL}} = \max\left[\frac{\hbar}{2\Delta E},\; \frac{\pi\hbar}{2\langle E\rangle}\right]
\]
For biological temperatures ($\Delta E \sim \kB T$), $\tau_{\mathrm{QSL}} \sim 10^{-13}$ s, relevant only for ultrafast molecular dynamics.

\paragraph{Signal-to-noise limit ($\tau_{\mathrm{SNR}}$).} Matched-filter detection requires integration time $\tau_{\mathrm{SNR}} = \rho_{\min} N_0/(2P_s)$ where $\rho_{\min} \sim 5$--10 is detection threshold, $N_0$ is noise power spectral density, and $P_s$ is signal power~\cite{poor1994}. For neural LFP signals, typical values yield $\tau_{\mathrm{SNR}} \sim 1$--10 ms.

\paragraph{Power limit ($\tau_{\mathrm{power}}$).} Dimensional collapse by $\Delta\mathcal{D}$ at dimensional chemical potential $\lambda_{\mathcal{D}} \sim \kB\Teff$ costs energy $\sim \lambda_{\mathcal{D}}\Delta\mathcal{D}$~\cite{todd2025biosystems}. At sustained power $P$, commit rate is bounded by
\[
\tau_{\mathrm{power}} = \frac{\lambda_{\mathcal{D}}\Delta\mathcal{D}}{P}
\]

\paragraph{Coherence time ($\tau_{\mathrm{coh}}$).} As derived above (Eq.~\ref{eq:coh-time}).

\subsection{The speed-flexibility trade-off}

The coherence time reveals opposing constraints on commit rate:

\begin{proposition}[Speed-Flexibility Frontier]\label{prop:tradeoff}
At fixed power $P$, systems face a Pareto frontier in $(M,r)$ space:
\begin{itemize}
\item \textbf{Increasing $M$}: Expands combinatorial flexibility (number of modules whose phases can vary independently) but slows commits via $\tau_{\mathrm{coh}} \propto (\mathrm{const})^{M}$
\item \textbf{Increasing $r$}: Speeds commits via reduced circular variance $(1-r)$ but restricts exploration to low-dimensional synchronized manifold
\end{itemize}
\end{proposition}

Proof: Taking logarithmic derivatives of Eq.~\ref{eq:coh-time}:
\begin{align}
\frac{\partial \ln\tau_{\mathrm{coh}}}{\partial M} &= \alpha(1-r)\ln\frac{2\pi}{\varepsilon} > 0 \\
\frac{\partial \ln\tau_{\mathrm{coh}}}{\partial r} &= -\alpha(M-1)\ln\frac{2\pi}{\varepsilon} < 0
\end{align}
State-space volume accessible in time $\tau$ scales as $\sim \exp(\Deff)$ where $\Deff \ge M$. Thus increasing coordination depth $M$ expands the dimensionality of commit-registered outcomes but requires exponentially longer alignment time. $\square$

\subsection{Parameter estimation protocols}

\paragraph{Coordination depth ($M$).} $M$ is the number of \textbf{degrees of freedom that must be simultaneously constrained} to satisfy the computation---not an arbitrary clustering choice. If a task requires coordinating visual motion (area MT) and shape (area V4), then $M = 2$ regardless of how many neurons are involved. If binding shape, motion, color, and spatial location, then $M = 4$. The coordination depth is set by the \emph{task structure}, not by the observer's clustering threshold.

\textbf{Operational estimation:}
\begin{enumerate}
\item Cluster channels/units by band-limited coherence (e.g., via phase-locking value $>$ threshold) or Granger causality into modules. Community detection algorithms such as Louvain~\cite{blondel2008} provide principled module identification from coherence matrices.
\item Identify the minimal set of modules whose joint activity predicts behavioral commits (e.g., via logistic regression on module order parameters). This identifies $M$ as the number of commit-relevant constraints.
\item Typical values: occipital visual tasks yield $M \sim 6$--10; cross-modal binding $M \sim 10$--15; default-mode wandering $M \sim 15$--20. These differences explain why complex integration is slower than simple detection.
\end{enumerate}

\paragraph{Kuramoto coherence ($r$).} Extract instantaneous phase $\theta_j(t)$ from bandpass-filtered LFP/EEG via Hilbert transform. Compute $r(t) = |\frac{1}{N}\sum_{j=1}^N e^{i\theta_j(t)}|$. Typical values: spontaneous cortex $\sim 0.2$--0.4; attention $\sim 0.5$--0.7; circadian networks $\sim 0.7$--0.9~\cite{fries2015}.

\paragraph{Phase-exploration rate ($\Delta\omega$).} Estimate from instantaneous frequency spread or phase diffusion, \emph{not} the carrier frequency. Practically: the standard deviation of the band-limited instantaneous frequency (from Hilbert or wavelet transform). Typical values: $\Delta\omega \sim 2\pi \times (5\text{--}20)$ rad/s for cortical gamma/alpha bands.

\section{Results}

\subsection{Visual perceptual binding windows}

Human visual perception exhibits temporal integration windows of 30--50 ms (flicker fusion at 20--30 Hz). We apply Eq.~\ref{eq:main-bound} with neural parameters.

\paragraph{Visual binding window.}
Take $M=10$ occipital modules (V1--V4, MT, posterior parietal sub-regions), $r=0.6$ (moderate attention~\cite{fries2015}), full tolerance $\varepsilon=\pi$ rad ($\pm 90^\circ$ across modules), and phase-exploration rate $\Delta\omega=2\pi\times 10$ rad/s (alpha-band spread). Using Eq.~\ref{eq:coh-time} with $\alpha=0.35$ (the validated modular-network value):
\[
\tau_{\mathrm{coh}} \approx \frac{1}{62.8}\left(\frac{2\pi}{\pi}\right)^{0.35(1-0.6)(9)}
= \frac{1}{62.8}\,2^{1.26}\approx \frac{2.4}{62.8}\approx 30\text{--}50~\mathrm{ms}.
\]

Computing all terms in Eq.~\ref{eq:main-bound}:
\begin{align}
\tau_{\mathrm{QSL}} &\sim 10^{-14} \,\mathrm{s} \quad \text{(negligible)} \\
\tau_{\mathrm{SNR}} &\sim 5\text{--}10~\mathrm{ms} \\
\tau_{\mathrm{power}} &\ll 1~\mathrm{ms} \quad \text{(per commit, local assembly power)} \\
\tau_{\mathrm{coh}} &\approx 30\text{--}50~\mathrm{ms}
\end{align}

The effective commit time is $\tau_{\mathrm{eff}} \approx 30$--$50~\mathrm{ms}$, matching human binding windows. \textbf{Coherence time dominates}. The calculation is order-of-magnitude; modest parameter variations (e.g., $r = 0.6$ vs $0.8$, $M = 6$ vs $10$) shift $\tau_{\mathrm{coh}}$ by factors of 2--5, consistent with observed inter-individual variability.

\subsection{Tachypsychia: dual-loop dissociation}

During acute stress or falls, subjects report subjective time slowing while objective reaction times remain unchanged~\cite{stetson2007}. This provides exclusion-based evidence about conscious experience:

\textbf{Exclusion logic:}
\begin{itemize}
\item Conscious time perception $\neq$ motor commit rate (time dilation without reaction time change)
\item Conscious time perception $\neq$ post-commit memory encoding (would also predict reaction time change)
\item \textbf{Inference}: Phenomenal temporal experience may correlate with continuous pre-commit coherent dynamics rather than discrete commit events
\end{itemize}

\textbf{Mechanistic explanation:} We propose dual commit pathways:

\emph{Perceptual loop} (cortical): High-$M$ ($\sim 10$--15 modules) coherent field dynamics across sensory and associative areas. Commits sparse (5--20 Hz), expensive. Arousal increases coherence $r$ and thus information rate $\mathcal{I}(t)$ via enhanced synchronization. Subjective duration scales as $T_{\mathrm{subjective}} \propto \int_0^{\Delta t} \mathcal{I}(t)\,dt$. Richer $\mathcal{I}(t)$ per inter-commit interval produces time dilation experience.

\emph{Motor loop} (cerebellar/basal ganglia): Low-$M$ ($\sim 3$--5 modules) primitives executing learned policies. Commits faster (50--150 ms), cheaper. Arousal modulates decision threshold/drift rate, preserving reaction time.

This is a mechanistic hypothesis consistent with the dissociation; alternative accounts exist (e.g., attentional sampling, memory density), but the key empirical discriminator is the predicted independence between temporal-order thresholds and simple RT under arousal.

\subsection{Alpha oscillations and temporal acuity}

Alpha oscillations (8--12 Hz) in visual cortex correlate with temporal acuity: individuals with faster alpha perceive time faster~\cite{samaha2015,cecere2015}. We interpret alpha as reflecting commit frequency: $r_{\mathrm{commit}} \sim f_\alpha$.

Prediction: Manipulating alpha via transcranial alternating current stimulation (tACS) should proportionally shift perceptual windows. Increasing $f_\alpha$ from 10 Hz to 12 Hz (20\% increase) predicts 20\% reduction in flicker fusion threshold (50 ms $\rightarrow$ 42 ms).

\subsection{Metabolic scaling across species}

Equation~\ref{eq:main-bound} predicts different regimes. When coordination time is slow ($\tau_{\mathrm{coh}} \gg \tau_{\mathrm{power}}$), coherence dominates (as in human perception). When power is limiting ($\tau_{\mathrm{power}} \gg \tau_{\mathrm{coh}}$), we obtain $\tau_{\mathrm{eff}} \sim 1/P$.

For animals with similar neural architecture ($M \sim 5$--10, $r \sim 0.5$--0.7), coordination time is roughly constant: $\tau_{\mathrm{coh}} \sim 10$--100 ms. However, power available for neural computation scales with mass-specific metabolic rate $P_{\mathrm{meta}}$. When $\tau_{\mathrm{power}}$ dominates (small animals with high metabolic rates), critical flicker fusion frequency is $f_{\mathrm{CFF}} \sim P/(\lambda_{\mathcal{D}}\Delta\mathcal{D}) \propto P_{\mathrm{meta}}$.

Healy et al.~\cite{healy2013} measured critical flicker fusion frequency across diverse taxa: flies $\sim 240$ Hz, humans $\sim 60$ Hz, leatherback turtles $\sim 15$ Hz. Log-log regression shows $R^2 \approx 0.6$ across three orders of magnitude in body mass, with $f_{\mathrm{CFF}} \propto P_{\mathrm{meta}}^{0.6}$. The framework predicts scaling in the observed direction; the exact exponent depends on how neural power allocation scales with whole-organism metabolic rate and on $\Delta\mathcal{D}$ across taxa.

\subsection{Parameter sensitivity enables dynamic range}

The exponential scaling in Eq.~\ref{eq:coh-time} produces extreme parameter sensitivity:
\begin{itemize}
\item $M: 5 \to 15$ shifts $\tau$ by $\sim 10^2$--$10^3$ (fixed $r = 0.6$, $\varepsilon = \pi$, $\alpha = 0.6$)
\item $r: 0.5 \to 0.7$ shifts $\tau$ by $\sim 10^1$--$10^2$ (fixed $M = 8$)
\item $\varepsilon: \pi \to \pi/3$ shifts $\tau$ by $\sim 10^1$--$10^2$
\end{itemize}

This is \emph{not a flaw} but the mechanism enabling massive dynamic range. Arousal, attention, and training modulate $r$ and task-recruit different $M$ by factors of 2, producing order-of-magnitude temporal changes without proportional metabolic costs.

\section{Discussion}

\subsection{Biological examples of the speed-flexibility trade-off}

\paragraph{Mind-wandering and creative insight (high $M$, low $r$).}
Default-mode network activity exhibits low inter-regional coherence ($r \sim 0.3$) and high coordination depth ($M \sim 12$--20 distributed modules). Commits are rare ($\sim 0.1$--1 Hz), producing subjective ``slow thinking'' but enabling novel cross-network associations unavailable to high-coherence focused states.

\paragraph{Focused attention (moderate $M$, moderate $r$).}
Attention increases coherence ($r: 0.3 \to 0.6$) and narrows coordination to task-relevant modules ($M: 15 \to 8$) by suppressing task-irrelevant networks~\cite{fries2015}. This produces faster commits (10--30 Hz) in a restricted ``spotlight'' subspace.

\paragraph{Automaticity and skill learning (low $M$, high $r$).}
Overlearned motor sequences compress to low-dimensional cerebellar/basal ganglia primitives ($M \sim 3$--5 coordinating modules) with tight forward-model coherence ($r \sim 0.8$). Commits are rapid (50--150 ms motor latencies) but inflexible.

\subsection{Testable predictions}

\paragraph{Alpha entrainment.}
tACS at alpha frequency should shift perceptual binding windows linearly with frequency change. Effect strongest in occipital cortex during visual tasks. 20\% frequency increase $\rightarrow$ 20\% threshold reduction.

\paragraph{Arousal effects on dual-loop dissociation.}
During simultaneous simple reaction time + temporal-order judgment under acute stress:
\begin{itemize}
\item Temporal-order discrimination threshold changes
\item Simple reaction time unchanged or slightly faster
\item Statistical independence between measures (evidence of separate pathways)
\end{itemize}

\paragraph{Coherence measures and consciousness.}
If phenomenal experience reflects pre-commit coherent dynamics, conscious states should correlate more strongly with phase-locking index, theta-gamma coupling, and beta-band coherence than with spike count or mean firing rate.

\subsection{Scope boundaries and limitations}

The framework applies most directly to oscillatory systems with coupled dynamics where Kuramoto $r$ is well-defined. Non-oscillatory systems may require alternative coherence measures.

\textbf{Limitations:}
\begin{itemize}
\item The independence assumption across modules is approximate; stronger inter-module coupling would reduce effective $M$.
\item Topology dependence: $\alpha$ varies with network structure. We find $\hat{\alpha} \approx 0.35$ for modular networks; other architectures yield different values (Table~\ref{tab:topology}).
\item Nonstationary $\Delta\omega$: During arousal or task transitions, phase-exploration rate may change dynamically.
\item Heavy-tailed first-passage times: Individual coherence events show high variance (resembling an Inverse Gaussian distribution); the formula describes median behavior. This is a \emph{feature}, not a bug: the heavy tail predicts that while median ``speed of thought'' is 30--50 ms, there will be occasional long-tail latencies ($>200$ ms) even in healthy processing, corresponding to ``lapses of attention.'' Concretely: trial-by-trial reaction time variability (RTV), a robust behavioral correlate of executive function~\cite{jensen2007}, should increase with task coordination depth (estimated $M$) and decrease with coherence (estimated $r$), even at constant mean RT.
\end{itemize}

\section{Conclusion}

We propose that coherence time sets a fundamental limit on the speed of distributed neural computation. The bound
\[
\tau_{\mathrm{eff}} = \max\left[\tau_{\mathrm{QSL}},\, \tau_{\mathrm{SNR}},\, \tau_{\mathrm{coh}},\, \tau_{\mathrm{power}}\right]
\]
with coherence time
\[
\tau_{\mathrm{coh}} = \frac{1}{\Delta\omega}\left(\frac{2\pi}{\varepsilon}\right)^{\alpha(1-r)(M-1)}
\]
unifies quantum, noise, and coordination constraints within a single framework. Numerical validation in modular Kuramoto networks supports exponential $M$-scaling ($\hat{\alpha} \approx 0.35$, $R^2 = 0.71$), with the formula's scope limited to hierarchically modular architectures---precisely the regime where biological neural networks operate. Direct empirical validation in neural systems remains an important direction for future work.

Key predictions:
\begin{enumerate}
\item \textbf{Speed-flexibility trade-off}: Increasing $M$ exponentially slows commits but expands combinatorial flexibility. Increasing $r$ speeds commits but restricts dynamics to low-dimensional synchronized manifolds.

\item \textbf{Dual-loop architecture}: Separate perceptual (high-$M$ cortical) and motor (low-$M$ cerebellar) pathways may explain tachypsychia dissociation.

\item \textbf{Parameter sensitivity mechanism}: Modest $r$ or $M$ shifts (factor of 2) produce order-of-magnitude temporal changes without proportional metabolic costs.

\item \textbf{Quantitative predictions}: Visual binding windows (20--50 ms), metabolic scaling, alpha entrainment linearity, dual-task dissociations under arousal.
\end{enumerate}

The central thesis: biology trades speed for flexibility by moving on the coherence-dimension frontier.

\section*{Data accessibility}

Simulation code and parameter estimation workflows available at \url{https://github.com/todd866/coherence-time-biosystems}.

\section*{Competing interests}

The author declares no competing interests.

\section*{Funding}

No external funding supported this work.

\section*{Acknowledgments}

The author thanks Abir Igamberdiev and anonymous reviewers at BioSystems for feedback on foundational frameworks.

\begin{thebibliography}{99}

\bibitem{miller2018}
Miller, E.K., Lundqvist, M., Bastos, A.M. (2018). Working memory 2.0. \textit{Neuron}, 100(2), 463--475.

\bibitem{fries2015}
Fries, P. (2015). Rhythms for cognition: communication through coherence. \textit{Neuron}, 88(1), 220--235.

\bibitem{singer1999}
Singer, W. (1999). Neuronal synchrony: a versatile code for the definition of relations? \textit{Neuron}, 24(1), 49--65.

\bibitem{lisman2013}
Lisman, J.E., Jensen, O. (2013). The theta-gamma neural code. \textit{Neuron}, 77(6), 1002--1016.

\bibitem{womelsdorf2007}
Womelsdorf, T., et al. (2007). Modulation of neuronal interactions through neuronal synchronization. \textit{Science}, 316(5831), 1609--1612.

\bibitem{engel2001}
Engel, A.K., Fries, P., Singer, W. (2001). Dynamic predictions: oscillations and synchrony in top-down processing. \textit{Nat. Rev. Neurosci.}, 2(10), 704--716.

\bibitem{varela2001}
Varela, F., et al. (2001). The brainweb: phase synchronization and large-scale integration. \textit{Nat. Rev. Neurosci.}, 2(4), 229--239.

\bibitem{fries2005}
Fries, P. (2005). A mechanism for cognitive dynamics: neuronal communication through neuronal coherence. \textit{Trends Cogn. Sci.}, 9(10), 474--480.

\bibitem{stetson2007}
Stetson, C., Fiesta, M.P., Eagleman, D.M. (2007). Does time really slow down during a frightening event? \textit{PLoS ONE}, 2(12), e1295.

\bibitem{healy2013}
Healy, K., McNally, L., Ruxton, G.D., Cooper, N., Jackson, A.L. (2013). Metabolic rate and body size are linked with perception of temporal information. \textit{Anim. Behav.}, 86(4), 685--696.

\bibitem{todd2025biosystems}
Todd, I. (2025). The limits of falsifiability: Dimensionality, measurement thresholds, and the sub-Landauer domain in biological systems. \textit{BioSystems}, 258, 105608.

\bibitem{todd2025demon}
Todd, I. (2025). Timing inaccessibility and the projection bound: Resolving Maxwell's demon for continuous biological substrates. \textit{BioSystems} (in press).

\bibitem{vernadsky1945}
Vernadsky, V.I. (1945). The biosphere and the noosphere. \textit{Am. Sci.}, 33(1), 1--12.

\bibitem{igamberdiev1985}
Igamberdiev, A.U. (1985). Time in biological systems. \textit{Zhurnal Obshchei Biologii}, 46(4), 471--482.

\bibitem{mandelstam1945}
Mandelstam, L., Tamm, I. (1945). The uncertainty relation between energy and time in non-relativistic quantum mechanics. \textit{J. Phys. (USSR)}, 9, 249--254.

\bibitem{margolus1998}
Margolus, N., Levitin, L.B. (1998). The maximum speed of dynamical evolution. \textit{Physica D}, 120(1--2), 188--195.

\bibitem{poor1994}
Poor, H.V. (1994). \textit{An Introduction to Signal Detection and Estimation} (2nd ed.). Springer-Verlag.

\bibitem{acebron2005}
Acebrón, J.A., Bonilla, L.L., Pérez Vicente, C.J., Ritort, F., Spigler, R. (2005). The Kuramoto model: A simple paradigm for synchronization phenomena. \textit{Rev. Mod. Phys.}, 77(1), 137--185.

\bibitem{pikovsky2001}
Pikovsky, A., Rosenblum, M., Kurths, J. (2001). \textit{Synchronization: A Universal Concept in Nonlinear Sciences}. Cambridge University Press.

\bibitem{mardia2000}
Mardia, K.V., Jupp, P.E. (2000). \textit{Directional Statistics}. John Wiley \& Sons.

\bibitem{ott2008}
Ott, E., Antonsen, T.M. (2008). Low dimensional behavior of large systems of globally coupled oscillators. \textit{Chaos}, 18(3), 037113.

\bibitem{watanabe1994}
Watanabe, S., Strogatz, S.H. (1994). Constants of motion for superconducting Josephson arrays. \textit{Physica D}, 74(3--4), 197--253.

\bibitem{cunningham2014}
Cunningham, J.P., Yu, B.M. (2014). Dimensionality reduction for large-scale neural recordings. \textit{Nat. Neurosci.}, 17(11), 1500--1509.

\bibitem{samaha2015}
Samaha, J., Postle, B.R. (2015). The speed of alpha-band oscillations predicts the temporal resolution of visual perception. \textit{Current Biology}, 25(22), 2985--2990.

\bibitem{cecere2015}
Cecere, R., Rees, G., Romei, V. (2015). Individual differences in alpha frequency drive crossmodal illusory perception. \textit{Current Biology}, 25(2), 231--235.

\bibitem{tononi2016}
Tononi, G., Boly, M., Massimini, M., Koch, C. (2016). Integrated information theory: from consciousness to its physical substrate. \textit{Nat. Rev. Neurosci.}, 17(7), 450--461.

\bibitem{dehaene2011}
Dehaene, S., Changeux, J.-P. (2011). Experimental and theoretical approaches to conscious processing. \textit{Neuron}, 70(2), 200--227.

\bibitem{friston2010}
Friston, K. (2010). The free-energy principle: a unified brain theory? \textit{Nat. Rev. Neurosci.}, 11(2), 127--138.

\bibitem{beggs2003}
Beggs, J.M., Plenz, D. (2003). Neuronal avalanches in neocortical circuits. \textit{J. Neurosci.}, 23(35), 11167--11177.

\bibitem{blondel2008}
Blondel, V.D., Guillaume, J.-L., Lambiotte, R., Lefebvre, E. (2008). Fast unfolding of communities in large networks. \textit{J. Stat. Mech.}, 2008(10), P10008.

\bibitem{jensen2007}
Jensen, A.R. (2007). Clocking the mind: Mental chronometry and individual differences. \textit{Elsevier}.

\bibitem{rosen1985}
Rosen, R. (1985). \textit{Anticipatory Systems: Philosophical, Mathematical and Methodological Foundations}. Pergamon Press.

\bibitem{matsuno1989}
Matsuno, K. (1989). \textit{Protobiology: Physical Basis of Biology}. CRC Press.

\end{thebibliography}

\end{document}
